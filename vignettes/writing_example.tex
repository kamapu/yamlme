% Options for packages loaded elsewhere
\PassOptionsToPackage{unicode}{hyperref}
\PassOptionsToPackage{hyphens}{url}
%
\documentclass[
]{article}
\usepackage{lmodern}
\usepackage{amssymb,amsmath}
\usepackage{ifxetex,ifluatex}
\ifnum 0\ifxetex 1\fi\ifluatex 1\fi=0 % if pdftex
  \usepackage[T1]{fontenc}
  \usepackage[utf8]{inputenc}
  \usepackage{textcomp} % provide euro and other symbols
\else % if luatex or xetex
  \usepackage{unicode-math}
  \defaultfontfeatures{Scale=MatchLowercase}
  \defaultfontfeatures[\rmfamily]{Ligatures=TeX,Scale=1}
\fi
% Use upquote if available, for straight quotes in verbatim environments
\IfFileExists{upquote.sty}{\usepackage{upquote}}{}
\IfFileExists{microtype.sty}{% use microtype if available
  \usepackage[]{microtype}
  \UseMicrotypeSet[protrusion]{basicmath} % disable protrusion for tt fonts
}{}
\makeatletter
\@ifundefined{KOMAClassName}{% if non-KOMA class
  \IfFileExists{parskip.sty}{%
    \usepackage{parskip}
  }{% else
    \setlength{\parindent}{0pt}
    \setlength{\parskip}{6pt plus 2pt minus 1pt}}
}{% if KOMA class
  \KOMAoptions{parskip=half}}
\makeatother
\usepackage{xcolor}
\IfFileExists{xurl.sty}{\usepackage{xurl}}{} % add URL line breaks if available
\IfFileExists{bookmark.sty}{\usepackage{bookmark}}{\usepackage{hyperref}}
\hypersetup{
  pdftitle={Test for yaml function},
  pdfauthor={Miguel Alvarez},
  hidelinks,
  pdfcreator={LaTeX via pandoc}}
\urlstyle{same} % disable monospaced font for URLs
\usepackage[margin=1in]{geometry}
\usepackage{color}
\usepackage{fancyvrb}
\newcommand{\VerbBar}{|}
\newcommand{\VERB}{\Verb[commandchars=\\\{\}]}
\DefineVerbatimEnvironment{Highlighting}{Verbatim}{commandchars=\\\{\}}
% Add ',fontsize=\small' for more characters per line
\usepackage{framed}
\definecolor{shadecolor}{RGB}{248,248,248}
\newenvironment{Shaded}{\begin{snugshade}}{\end{snugshade}}
\newcommand{\AlertTok}[1]{\textcolor[rgb]{0.94,0.16,0.16}{#1}}
\newcommand{\AnnotationTok}[1]{\textcolor[rgb]{0.56,0.35,0.01}{\textbf{\textit{#1}}}}
\newcommand{\AttributeTok}[1]{\textcolor[rgb]{0.77,0.63,0.00}{#1}}
\newcommand{\BaseNTok}[1]{\textcolor[rgb]{0.00,0.00,0.81}{#1}}
\newcommand{\BuiltInTok}[1]{#1}
\newcommand{\CharTok}[1]{\textcolor[rgb]{0.31,0.60,0.02}{#1}}
\newcommand{\CommentTok}[1]{\textcolor[rgb]{0.56,0.35,0.01}{\textit{#1}}}
\newcommand{\CommentVarTok}[1]{\textcolor[rgb]{0.56,0.35,0.01}{\textbf{\textit{#1}}}}
\newcommand{\ConstantTok}[1]{\textcolor[rgb]{0.00,0.00,0.00}{#1}}
\newcommand{\ControlFlowTok}[1]{\textcolor[rgb]{0.13,0.29,0.53}{\textbf{#1}}}
\newcommand{\DataTypeTok}[1]{\textcolor[rgb]{0.13,0.29,0.53}{#1}}
\newcommand{\DecValTok}[1]{\textcolor[rgb]{0.00,0.00,0.81}{#1}}
\newcommand{\DocumentationTok}[1]{\textcolor[rgb]{0.56,0.35,0.01}{\textbf{\textit{#1}}}}
\newcommand{\ErrorTok}[1]{\textcolor[rgb]{0.64,0.00,0.00}{\textbf{#1}}}
\newcommand{\ExtensionTok}[1]{#1}
\newcommand{\FloatTok}[1]{\textcolor[rgb]{0.00,0.00,0.81}{#1}}
\newcommand{\FunctionTok}[1]{\textcolor[rgb]{0.00,0.00,0.00}{#1}}
\newcommand{\ImportTok}[1]{#1}
\newcommand{\InformationTok}[1]{\textcolor[rgb]{0.56,0.35,0.01}{\textbf{\textit{#1}}}}
\newcommand{\KeywordTok}[1]{\textcolor[rgb]{0.13,0.29,0.53}{\textbf{#1}}}
\newcommand{\NormalTok}[1]{#1}
\newcommand{\OperatorTok}[1]{\textcolor[rgb]{0.81,0.36,0.00}{\textbf{#1}}}
\newcommand{\OtherTok}[1]{\textcolor[rgb]{0.56,0.35,0.01}{#1}}
\newcommand{\PreprocessorTok}[1]{\textcolor[rgb]{0.56,0.35,0.01}{\textit{#1}}}
\newcommand{\RegionMarkerTok}[1]{#1}
\newcommand{\SpecialCharTok}[1]{\textcolor[rgb]{0.00,0.00,0.00}{#1}}
\newcommand{\SpecialStringTok}[1]{\textcolor[rgb]{0.31,0.60,0.02}{#1}}
\newcommand{\StringTok}[1]{\textcolor[rgb]{0.31,0.60,0.02}{#1}}
\newcommand{\VariableTok}[1]{\textcolor[rgb]{0.00,0.00,0.00}{#1}}
\newcommand{\VerbatimStringTok}[1]{\textcolor[rgb]{0.31,0.60,0.02}{#1}}
\newcommand{\WarningTok}[1]{\textcolor[rgb]{0.56,0.35,0.01}{\textbf{\textit{#1}}}}
\usepackage{graphicx}
\makeatletter
\def\maxwidth{\ifdim\Gin@nat@width>\linewidth\linewidth\else\Gin@nat@width\fi}
\def\maxheight{\ifdim\Gin@nat@height>\textheight\textheight\else\Gin@nat@height\fi}
\makeatother
% Scale images if necessary, so that they will not overflow the page
% margins by default, and it is still possible to overwrite the defaults
% using explicit options in \includegraphics[width, height, ...]{}
\setkeys{Gin}{width=\maxwidth,height=\maxheight,keepaspectratio}
% Set default figure placement to htbp
\makeatletter
\def\fps@figure{htbp}
\makeatother
\setlength{\emergencystretch}{3em} % prevent overfull lines
\providecommand{\tightlist}{%
  \setlength{\itemsep}{0pt}\setlength{\parskip}{0pt}}
\setcounter{secnumdepth}{-\maxdimen} % remove section numbering

\title{Test for yaml function}
\author{Miguel Alvarez}
\date{}

\begin{document}
\maketitle

\hypertarget{introduction}{%
\section{Introduction}\label{introduction}}

The package \href{https://github.com/kamapu/yamlme}{\texttt{yamlme}} is
developed to enhance automatic generation of reports and documents from
different packages, for instance reporting data sources by the package
\href{https://github.com/kamapu/vegtable}{\texttt{vegtable}} or
producing check-lists by the package
\href{https://github.com/kamapu/taxlist}{\texttt{taxlist}}. These
applications are at the moment tested in an experimental way, for
instance in the package
\href{https://github.com/kamapu/vegtable2}{\texttt{vegtable2}} (see
function \texttt{report\_communities()}).

The function \texttt{write\_yaml()} is at the moment implemented in an
own package. For details, a brief description will be retrieved by
\texttt{?write\_yaml}.

\begin{Shaded}
\begin{Highlighting}[]
\NormalTok{devtools}\OperatorTok{::}\KeywordTok{install\_github}\NormalTok{(}\StringTok{"kamapu/yamlme"}\NormalTok{)}
\KeywordTok{library}\NormalTok{(yamlme)}
\KeywordTok{library}\NormalTok{(rmarkdown)}
\end{Highlighting}
\end{Shaded}

\hypertarget{state-of-the-art}{%
\section{State of the Art}\label{state-of-the-art}}

The idea behind this function is to create Rmd documents with conten by
using R-code. For instance a PDF document by setting the different yaml
items as arguments in the function.

\begin{Shaded}
\begin{Highlighting}[]
\NormalTok{Docu \textless{}{-}}\StringTok{ }\KeywordTok{write\_yaml}\NormalTok{(}
        \DataTypeTok{title=}\StringTok{"First Function Version"}\NormalTok{,}
        \DataTypeTok{author=}\StringTok{"Miguel Alvarez"}\NormalTok{,}
        \DataTypeTok{output=}\StringTok{"pdf\_document"}\NormalTok{)}
\KeywordTok{cat}\NormalTok{(Docu)}
\end{Highlighting}
\end{Shaded}

\begin{verbatim}
## ---
##  title: First Function Version
##  author: Miguel Alvarez
##  output: pdf_document
##  
## ---
## 
## 
\end{verbatim}

The names of the yaml entries are provided by the user, except for the
defined arguments \texttt{append}, \texttt{body}, and \texttt{filename},
which I am assuming are not required in Rmarkdown documents.

To produce a full document, we need to include a body part, an output
file and render it with rmarkdown.

\begin{Shaded}
\begin{Highlighting}[]
\NormalTok{Docu \textless{}{-}}\StringTok{ }\KeywordTok{write\_yaml}\NormalTok{(}
        \DataTypeTok{title=}\StringTok{"First Function Version"}\NormalTok{,}
        \DataTypeTok{author=}\StringTok{"Miguel Alvarez"}\NormalTok{,}
        \DataTypeTok{output=}\StringTok{"pdf\_document"}\NormalTok{,}
        \DataTypeTok{body=}\StringTok{"This is a first incursion in Rmarkdown."}\NormalTok{,}
        \DataTypeTok{filename=}\StringTok{"test\_doc.Rmd"}\NormalTok{)}
\KeywordTok{render}\NormalTok{(}\StringTok{"test\_doc.Rmd"}\NormalTok{)}
\end{Highlighting}
\end{Shaded}

\textbf{THE IDEA} is to insert the entries by a combination of 1)
character vectors of length 1 (the previous examples), 2) character
vectors longer than 1 and 3) complex structures by lists. At the moment
only alternatives 1 and 2 are implemented, while the challenge with more
complex entries is to make the formatting in a recursive way and at the
same time detect the hierarchical rank (depth) to properly set
indentation.

\begin{Shaded}
\begin{Highlighting}[]
\NormalTok{Docu \textless{}{-}}\StringTok{ }\KeywordTok{write\_yaml}\NormalTok{(}
        \DataTypeTok{title=}\StringTok{"First Function Version"}\NormalTok{,}
        \DataTypeTok{author=}\StringTok{"Miguel Alvarez"}\NormalTok{,}
        \DataTypeTok{output=}\StringTok{"pdf\_document"}\NormalTok{,}
        \StringTok{"header{-}includes"}\NormalTok{=}\KeywordTok{c}\NormalTok{(}
                \StringTok{"{-} }\CharTok{\textbackslash{}\textbackslash{}}\StringTok{usepackage[utf8]\{inputenc\}"}\NormalTok{,}
                \StringTok{"{-} }\CharTok{\textbackslash{}\textbackslash{}}\StringTok{usepackage[T1]\{fontenc\}"}\NormalTok{),}
        \DataTypeTok{append=}\StringTok{"\# Document written with \textquotesingle{}write\_yaml()\textquotesingle{}"}\NormalTok{,}
        \DataTypeTok{body=}\StringTok{"This is a first incursion in Rmarkdown."}\NormalTok{,}
        \DataTypeTok{filename=}\StringTok{"test\_doc.Rmd"}\NormalTok{)}
\end{Highlighting}
\end{Shaded}

Note that the name of the entry \textbf{header-includes} have to be
quoted because of the dash, while back-slashes have to be escaped.

\hypertarget{desiderata}{%
\section{Desiderata}\label{desiderata}}

The next step should be to implement lists as arguments for hierarchical
entries in the yaml head. For instance

\begin{Shaded}
\begin{Highlighting}[]
\NormalTok{output=}\KeywordTok{list}\NormalTok{(}
        \DataTypeTok{pdf\_file=}\StringTok{"default"}\NormalTok{)}
\end{Highlighting}
\end{Shaded}

should write

\begin{verbatim}
---
output:
  pdf_document: default
---
\end{verbatim}

A more complex example can be provided, for instance, by multiple
outputs

\begin{verbatim}
---
output:
  html_document:
    toc: true
    toc_float: true
  word_document:
    fig_caption: false
---
\end{verbatim}

In that case, the argument output should be

\begin{Shaded}
\begin{Highlighting}[]
\NormalTok{output=}\KeywordTok{list}\NormalTok{(}
        \DataTypeTok{html\_document=}\KeywordTok{list}\NormalTok{(}
                \DataTypeTok{toc=}\StringTok{"true"}\NormalTok{,}
                \DataTypeTok{toc\_float=}\StringTok{"true"}\NormalTok{),}
        \DataTypeTok{word\_document=}\KeywordTok{list}\NormalTok{(}
                \DataTypeTok{fig\_caption=}\StringTok{"false"}\NormalTok{))}
\end{Highlighting}
\end{Shaded}

The function is expected to generate automatically the structure of the
head entry depending on the structure of the list in the function. The
good news is that we can tweak the (current) limitations of
\texttt{write\_yaml()} by using the argument \texttt{append}.

\begin{Shaded}
\begin{Highlighting}[]
\NormalTok{Docu \textless{}{-}}\StringTok{ }\KeywordTok{write\_yaml}\NormalTok{(}
        \DataTypeTok{title=}\StringTok{"First Function Version"}\NormalTok{,}
        \DataTypeTok{author=}\StringTok{"Miguel Alvarez"}\NormalTok{,}
        \DataTypeTok{append=}\KeywordTok{paste}\NormalTok{(}\KeywordTok{c}\NormalTok{(}
                        \StringTok{"\# Appended code"}\NormalTok{,}
                        \StringTok{"output:"}\NormalTok{,}
                        \StringTok{"  html\_document:"}\NormalTok{,}
                        \StringTok{"    toc: true"}\NormalTok{,}
                        \StringTok{"    toc\_float: true"}\NormalTok{,}
                        \StringTok{"  word\_document:"}\NormalTok{,}
                        \StringTok{"    fig\_caption: false"}\NormalTok{),}
                \DataTypeTok{collapse=}\StringTok{"}\CharTok{\textbackslash{}n}\StringTok{"}\NormalTok{),}
        \DataTypeTok{body=}\KeywordTok{paste}\NormalTok{(}\KeywordTok{c}\NormalTok{(}
                        \StringTok{"\# Breakfast"}\NormalTok{,}
                        \StringTok{""}\NormalTok{,}
                        \StringTok{"Just a cup of coffee."}\NormalTok{,}
                        \StringTok{""}\NormalTok{,}
                        \StringTok{"\# Midday"}\NormalTok{,}
                        \StringTok{""}\NormalTok{,}
                        \StringTok{"{-} A pizza from the Botan Grill"}\NormalTok{,}
                        \StringTok{"{-} A bottle of coca{-}cola"}\NormalTok{),}
                \DataTypeTok{collapse=}\StringTok{"}\CharTok{\textbackslash{}n}\StringTok{"}\NormalTok{),}
        \DataTypeTok{filename=}\StringTok{"test\_doc.Rmd"}\NormalTok{)}
\KeywordTok{render}\NormalTok{(}\StringTok{"test\_doc.Rmd"}\NormalTok{, }\DataTypeTok{output\_format=}\StringTok{"all"}\NormalTok{)}
\end{Highlighting}
\end{Shaded}

Some possible improvements may include:

\begin{itemize}
\tightlist
\item
  Output object could be defined as ``S3'' class.
\item
  Optionally no object will be produced, just a file.
\item
  Other alternative is to produce a summarized print in the console,
  when object is not assigned to an object.
\end{itemize}

\hypertarget{motivation}{%
\section{Motivation}\label{motivation}}

Why to do this effort? Our intention is to use the function for the
``automatic'' generation of reports. For instance, we can define an
``alias function'' including a template of a document.

\begin{Shaded}
\begin{Highlighting}[]
\NormalTok{auto\_report \textless{}{-}}\StringTok{ }\ControlFlowTok{function}\NormalTok{(}\DataTypeTok{title=}\StringTok{"Automatic Document"}\NormalTok{,}
        \DataTypeTok{author=}\StringTok{"Me Robot"}\NormalTok{,}
        \DataTypeTok{output=}\StringTok{"pdf\_document"}\NormalTok{,}
        \DataTypeTok{body=}\StringTok{"Hello world!"}\NormalTok{,}
        \DataTypeTok{filename=}\StringTok{"test\_doc.Rmd"}\NormalTok{, ...) \{}
    \KeywordTok{write\_yaml}\NormalTok{(}
            \DataTypeTok{title=}\NormalTok{title,}
            \DataTypeTok{author=}\NormalTok{author,}
            \DataTypeTok{output=}\NormalTok{output,}
            \DataTypeTok{body=}\NormalTok{body,}
            \DataTypeTok{filename=}\NormalTok{filename,}
\NormalTok{            ...)}
\NormalTok{\}}
\NormalTok{Docu \textless{}{-}}\StringTok{ }\KeywordTok{auto\_report}\NormalTok{()}
\KeywordTok{render}\NormalTok{(}\StringTok{"test\_doc.Rmd"}\NormalTok{)}
\end{Highlighting}
\end{Shaded}

We can then use this function to customize the template by changing the
values of the arguments and even adding new ones (I guess, it is not
possible to suppress yaml entries).

\begin{Shaded}
\begin{Highlighting}[]
\NormalTok{Docu \textless{}{-}}\StringTok{ }\KeywordTok{auto\_report}\NormalTok{(}
        \DataTypeTok{title=}\StringTok{"Human Report"}\NormalTok{,}
        \DataTypeTok{author=}\StringTok{"Bisrat"}\NormalTok{,}
        \DataTypeTok{date=}\StringTok{"1.1.2013"}\NormalTok{,}
        \DataTypeTok{output=}\StringTok{"html\_document"}\NormalTok{,}
        \DataTypeTok{body=}\KeywordTok{paste}\NormalTok{(}\KeywordTok{c}\NormalTok{(}
                        \StringTok{"\# Breakfast"}\NormalTok{,}
                        \StringTok{""}\NormalTok{,}
                        \StringTok{"Just a cup of coffee."}\NormalTok{,}
                        \StringTok{""}\NormalTok{,}
                        \StringTok{"\# Midday"}\NormalTok{,}
                        \StringTok{""}\NormalTok{,}
                        \StringTok{"{-} A pizza from the Botan Grill"}\NormalTok{,}
                        \StringTok{"{-} A bottle of coca{-}cola"}\NormalTok{),}
                \DataTypeTok{collapse=}\StringTok{"}\CharTok{\textbackslash{}n}\StringTok{"}\NormalTok{))}
\KeywordTok{render}\NormalTok{(}\StringTok{"test\_doc.Rmd"}\NormalTok{)}
\end{Highlighting}
\end{Shaded}

Enjoy!

\end{document}
